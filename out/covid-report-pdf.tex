% Options for packages loaded elsewhere
\PassOptionsToPackage{unicode}{hyperref}
\PassOptionsToPackage{hyphens}{url}
%
\documentclass[
]{article}
\usepackage{lmodern}
\usepackage{amssymb,amsmath}
\usepackage{ifxetex,ifluatex}
\ifnum 0\ifxetex 1\fi\ifluatex 1\fi=0 % if pdftex
  \usepackage[T1]{fontenc}
  \usepackage[utf8]{inputenc}
  \usepackage{textcomp} % provide euro and other symbols
\else % if luatex or xetex
  \usepackage{unicode-math}
  \defaultfontfeatures{Scale=MatchLowercase}
  \defaultfontfeatures[\rmfamily]{Ligatures=TeX,Scale=1}
\fi
% Use upquote if available, for straight quotes in verbatim environments
\IfFileExists{upquote.sty}{\usepackage{upquote}}{}
\IfFileExists{microtype.sty}{% use microtype if available
  \usepackage[]{microtype}
  \UseMicrotypeSet[protrusion]{basicmath} % disable protrusion for tt fonts
}{}
\makeatletter
\@ifundefined{KOMAClassName}{% if non-KOMA class
  \IfFileExists{parskip.sty}{%
    \usepackage{parskip}
  }{% else
    \setlength{\parindent}{0pt}
    \setlength{\parskip}{6pt plus 2pt minus 1pt}}
}{% if KOMA class
  \KOMAoptions{parskip=half}}
\makeatother
\usepackage{xcolor}
\IfFileExists{xurl.sty}{\usepackage{xurl}}{} % add URL line breaks if available
\IfFileExists{bookmark.sty}{\usepackage{bookmark}}{\usepackage{hyperref}}
\hypersetup{
  pdftitle={COVID-19 cases and deaths},
  pdfauthor={Attila Szuts},
  hidelinks,
  pdfcreator={LaTeX via pandoc}}
\urlstyle{same} % disable monospaced font for URLs
\usepackage[margin=1in]{geometry}
\usepackage{longtable,booktabs}
% Correct order of tables after \paragraph or \subparagraph
\usepackage{etoolbox}
\makeatletter
\patchcmd\longtable{\par}{\if@noskipsec\mbox{}\fi\par}{}{}
\makeatother
% Allow footnotes in longtable head/foot
\IfFileExists{footnotehyper.sty}{\usepackage{footnotehyper}}{\usepackage{footnote}}
\makesavenoteenv{longtable}
\usepackage{graphicx,grffile}
\makeatletter
\def\maxwidth{\ifdim\Gin@nat@width>\linewidth\linewidth\else\Gin@nat@width\fi}
\def\maxheight{\ifdim\Gin@nat@height>\textheight\textheight\else\Gin@nat@height\fi}
\makeatother
% Scale images if necessary, so that they will not overflow the page
% margins by default, and it is still possible to overwrite the defaults
% using explicit options in \includegraphics[width, height, ...]{}
\setkeys{Gin}{width=\maxwidth,height=\maxheight,keepaspectratio}
% Set default figure placement to htbp
\makeatletter
\def\fps@figure{htbp}
\makeatother
\setlength{\emergencystretch}{3em} % prevent overfull lines
\providecommand{\tightlist}{%
  \setlength{\itemsep}{0pt}\setlength{\parskip}{0pt}}
\setcounter{secnumdepth}{5}

\title{COVID-19 cases and deaths}
\author{Attila Szuts}
\date{28/11/2020}

\begin{document}
\maketitle

{
\setcounter{tocdepth}{2}
\tableofcontents
}
\newpage

\hypertarget{executive-summary}{%
\section{Executive summary}\label{executive-summary}}

In this report I am going to summarise my main findings on the pattern
of association between confirmed COVID-19 cases and deaths on a given
day on a country level. I used a simple linear regression with log
transformed predictor and outcome variables. I found a clear linear
relationship between the log of number of confirmed cases and log of
number of deaths: countries that have one percent higher confirmed
cases, have roughly one percent higher deaths, on average
\((\beta=1.031)\); however causal relationship has not been investigated
in any way. The selected model can explain 89\% of the variance.
\((R^2=0.8859)\) This suggests that in general as the number of cases
increase the number of deaths increase in the same pace, i.e.~very
broadly speaking high and low number of confirmed cases have the same
fraction of deaths. Adding further predictor variables (like population
density and response to pandemic situation) could help us better predic
the number of cases and explain outliers better.

\hypertarget{introduction}{%
\section{Introduction}\label{introduction}}

My main variables are countries, confirmed cases, deaths and total
population of a country. Numbers on confirmed cases and deaths were
collected by the Center for Systems Science and Engineering at Johns
Hopkins University. Population data for each country was downloaded from
the World Bank's site. The population of my analysis are the world's
countries during the pandemic and I am analysing a cross-sectional
subset on 2020-09-10. Data quality can vary from country to country
depending on the countries' healthcare system's level of sophistication.

\hypertarget{histogram-and-summary-statistics}{%
\section{Histogram and summary
statistics}\label{histogram-and-summary-statistics}}

In the histograms below we can see all of our variables are skewed to
the left, having a long right tail. These extreme values are not
measurement errors, but are in fact countries with extremely high
numbers of COVID-19 cases. These countries include Brazil, India, and
the United states with the top 3 number of confirmed cases; Brazil,
India and Mexico with the top 3 number of deaths. These extreme values
could be due to a number of factors, including higher than average
population, population density and lack of counter-pandemic measures.

\includegraphics{covid-report-pdf_files/figure-latex/unnamed-chunk-2-1.pdf}

\hypertarget{variable-transformations}{%
\section{Variable transformations}\label{variable-transformations}}

Since we would like to model percentage differences in this analysis
(for countries that have higher number of cases, how will the average
number of deaths change), it would make sense to transform our \(x\) and
\(y\) variables logarithmically (if it also makes sense from a
statistical point of view). To investigate this question, I plotted my
data on a scatterplot and used the previous histograms and summary
statistics to establish, that indeed the clearest linear pattern of
association emerges when both the predictor and outcome variable is
transformed. Also, my data is cross-sectional and taking logs of
variables we can solve the lack of baseline for comparison.

Because of log transformations, I had to exclude countries, where the
number of confirmed cases or deaths were 0. There were no countries that
did not have at least a single case, but there were 11 countries for
example Mongoloia, Bhutan and Cambodia that all had 0 deaths.

\hypertarget{model-of-choice-simple-linear-regression}{%
\section{Model of choice: simple linear
regression}\label{model-of-choice-simple-linear-regression}}

In my analysis I used different models to try to uncover the pattern of
association between confirmed cases and deaths and to try to find the
model that can explain the most variance in my data while also being
easily interpreted. Because of these reasons I chose the simple linear
regression that I am going to introduce in the following section.

The formula for this model is:

\[ln(\text{deaths}) = \alpha + \beta \times ln(\text{cases})\]
\[\alpha = -4.272\] \[\beta = 1.031\] These parameters can be
interpreted as the number of deaths is higher 1.031\% on average for
observations with 1\% higher confirmed cases. The intercept (\(\alpha\))
means, that when there is only one confirmed case the average of the log
of deaths is -4.272 (which is meaningless in this context).

\hypertarget{hypothesis-testing-and-residual-analysis}{%
\section{Hypothesis testing and residual
analysis}\label{hypothesis-testing-and-residual-analysis}}

Next, I would like to investigate, if the true \(\beta\) is 0 or not, or
said otherwise: if there is a true relationship between our \(x\) and
\(y\). So for this I set up the following hypothesis test:
\[H_0 : \beta_{true}=0, \space H_A : \beta_{true}\neq 0\]

The estimated t-statistics is 35.94, with p-value: 8.63e-81, which means
our test is significant at 1\%. Thus I reject the \(H_0\), which means
the confirmed cases is not uncorrelated with the number of deaths.

Next, I investigated the residuals:

The largest negative deviance from the predicted value is found in
Singapore with predicted number of deaths of 1139, but the real value is
only 27. This means, that in Singapore the situation is better, than the
model would predict based on the number of active cases. The largest
positive deviance from the predicted value is found in Yemen with
predicted number of deaths 36, but the real value is 593. This means,
that in Yemen, the situation is worse, than the model would predict
based on the number of active cases.

\hypertarget{appendix}{%
\section{Appendix}\label{appendix}}

\hypertarget{transforming-variables}{%
\subsection{Transforming variables}\label{transforming-variables}}

\includegraphics{covid-report-pdf_files/figure-latex/unnamed-chunk-6-1.pdf}
\includegraphics{covid-report-pdf_files/figure-latex/unnamed-chunk-6-2.pdf}
\includegraphics{covid-report-pdf_files/figure-latex/unnamed-chunk-6-3.pdf}
\includegraphics{covid-report-pdf_files/figure-latex/unnamed-chunk-6-4.pdf}

\newpage

\hypertarget{estimating-different-models}{%
\subsection{Estimating different
models}\label{estimating-different-models}}

You can see the different models, that I used in this table. All models
use log transformed variables. First, I built a simple linear
regression. Then I created a quadratic linear regression, but it did not
yield better results at all. Creating a Piecewise Linear Splines model
did not prove to be better in terms of explanatory power either.
Finally, I built an OLS weighted by population which did yield better
\(R^2\), however not much more.

In the table we can see that \(\beta\)-s are all very similar. For the
quadratic linear model it is slightly more difficult to interpret the
results, as you need to get the first derivative to be able to quantify
and interpret the relationship between confirmed cases and deaths. We
can see from \(\beta_2\) that the parabola is convex \((\beta_2>0)\).
Countries which are one unit larger than the average of \(x\) (confirmed
cases) are higher by \(\gamma=\beta_1+2\beta_2\bar x\) in \(y\) (deaths)
on average.

The PLS model can be interpreted as there is a 1\% increase in deaths on
average for countries that have 1\% confirmed cases if the total number
of cases is less than \ensuremath{3.390621\times 10^{5}} and there is a
1.21\% increase when the total number of cases is larger than or equal
to \ensuremath{3.390621\times 10^{5}}

The weighted OLS model can be interpreted exactly the same as the simple
linear regression, except that observations are weighted by their
population. Results are also very similar, except for having twice as
big \(SE\) for \(\beta\).

I chose the simple linear regression because it is very similar in terms
of \(R^2\), \(\beta\) and \(SE(\beta)\) to the other three models, but
it is the simplest. Following the law of parsimony, it makes sense to
choose the simplest model, that can explain the most of the variance,
hence my choice for the simple linear regression. Altough it is worth
mentioning, that the PLS shows an interesting insight that could be
worth further exploration. Namely, after the cutoff, there was a slight
increase in \(\beta\), altough in \(SE(\beta)\) as well. This could be
interesting to investigate, if countries with higher cases, are
vulnerable to more deaths. Which could make sense, considering the
consequences of higher number of cases: higher risk of further
transmission, less capacity in healthcare facilities, etc. The WOLS
model also could be useful in further analysis, since we can clearly
see, that population can be a useful asset in our model, which again
makes sense. Countries with more people tend to have more crowded cities
which can lead to higher infection rates. However, it would probably be
more straightforward to account for these kinds of transmission trends
instead of weighing countries simply based on population.

\begin{table}[h]
\begin{center}
\begin{small}
\begin{tabular}{l c c c c}
\hline
 & Simple linear & Quadratic linear & PLS & Weighted OLS \\
\hline
Intercept             & $-4.27^{***}$ & $-2.22^{*}$  & $-4.10^{***}$ & $-3.08^{***}$ \\
                      & $(0.30)$      & $(0.85)$     & $(0.35)$      & $(0.78)$      \\
ln(cases)             & $1.03^{***}$  & $0.59^{***}$ &               & $0.95^{***}$  \\
                      & $(0.03)$      & $(0.17)$     &               & $(0.06)$      \\
ln(cases)$^2$         &               & $0.02^{**}$  &               &               \\
                      &               & $(0.01)$     &               &               \\
ln(cases$<$339062.1)  &               &              & $1.01^{***}$  &               \\
                      &               &              & $(0.04)$      &               \\
ln(cases$>$=339062.1) &               &              & $1.21^{***}$  &               \\
                      &               &              & $(0.13)$      &               \\
\hline
R$^2$                 & $0.89$        & $0.89$       & $0.89$        & $0.93$        \\
Adj. R$^2$            & $0.89$        & $0.89$       & $0.89$        & $0.93$        \\
Num. obs.             & $170$         & $170$        & $170$         & $170$         \\
RMSE                  & $0.83$        & $0.82$       & $0.83$        & $4315.62$     \\
\hline
\multicolumn{5}{l}{\tiny{$^{***}p<0.001$; $^{**}p<0.01$; $^{*}p<0.05$}}
\end{tabular}
\end{small}
\caption{Modelling case fatality and confirmed COVID-19 cases}
\label{table:coefficients}
\end{center}
\end{table}
\newpage

\includegraphics{covid-report-pdf_files/figure-latex/unnamed-chunk-8-1.pdf}
\includegraphics{covid-report-pdf_files/figure-latex/unnamed-chunk-8-2.pdf}
\includegraphics{covid-report-pdf_files/figure-latex/unnamed-chunk-8-3.pdf}
\includegraphics{covid-report-pdf_files/figure-latex/unnamed-chunk-8-4.pdf}

\newpage

\hypertarget{investigating-biggest-residuals}{%
\subsection{Investigating biggest
residuals}\label{investigating-biggest-residuals}}

\begin{longtable}[]{@{}lrrrr@{}}
\caption{Countries with higher predicted than actual
deaths}\tabularnewline
\toprule
Country & Confirmed cases & Actual deaths & Predicted deaths &
Residual\tabularnewline
\midrule
\endfirsthead
\toprule
Country & Confirmed cases & Actual deaths & Predicted deaths &
Residual\tabularnewline
\midrule
\endhead
Singapore & 57859 & 27 & 1139 & -3.742357\tabularnewline
Qatar & 127600 & 219 & 2576 & -2.464849\tabularnewline
Burundi & 515 & 1 & 9 & -2.168331\tabularnewline
Sri Lanka & 4523 & 13 & 82 & -1.844374\tabularnewline
Iceland & 3373 & 10 & 61 & -1.804153\tabularnewline
\bottomrule
\end{longtable}

\begin{longtable}[]{@{}lrrrr@{}}
\caption{Countries with lower predicted than actual
deaths}\tabularnewline
\toprule
Country & Confirmed cases & Actual deaths & Predicted deaths &
Residual\tabularnewline
\midrule
\endfirsthead
\toprule
Country & Confirmed cases & Actual deaths & Predicted deaths &
Residual\tabularnewline
\midrule
\endhead
Yemen & 2051 & 593 & 36 & 2.791553\tabularnewline
Italy & 343770 & 36111 & 7159 & 1.618238\tabularnewline
Mexico & 810020 & 83497 & 17329 & 1.572451\tabularnewline
Ecuador & 145848 & 12175 & 2957 & 1.415357\tabularnewline
Chad & 1274 & 90 & 22 & 1.397286\tabularnewline
\bottomrule
\end{longtable}

\end{document}
